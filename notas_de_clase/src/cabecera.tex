\documentclass[spanish, letterpaper, 12pt, oneside]{book}
\usepackage{babel}
\usepackage[utf8]{inputenc}
\usepackage{babelbib}
\usepackage{etoolbox}
\usepackage{amstext, amssymb, amsthm, amsmath, amsbsy}
\usepackage{tabularx}
\usepackage{caption}
\usepackage{graphicx}
\usepackage{cancel}
\usepackage{url}
\usepackage[final]{pdfpages}
\usepackage{geometry}
\usepackage{enumerate}
\usepackage{mathtools}
\usepackage{subfig}
\usepackage{subfiles}	
\usepackage{hyperref} %% Para incluir detalles cucas del pdf
\usepackage{float}
\usepackage{multirow}
\usepackage{leftidx}
\usepackage{cite}  %% Para poner bonitas las citas
\usepackage{bookmark} %% Para poder organizar las etiquetas en pdf
\usepackage{cleveref}


%% Configuracion paquetes

\patchcmd\btxselectlanguage{\csname}{\csname TEMPPATCH}{}{} % Para hacer a babelbib funcionar


\geometry{
	verbose
}

\hypersetup{
	bookmarksopen=true,
	colorlinks=true,
	linkcolor=blue,
	citecolor=blue,
	urlcolor=black,	
	linktoc=all,
	bookmarksopen=false,
	pdftitle={Notas de Clase: Mecánica de los Medios Continuos},
	pdfauthor={Mecánica Aplicada},
	pdfkeywords={Mecánica de los medios continuos, Mecánica de sólidos},
	pdfsubject={Notas de Clase},
	pdfpagemode=UseOutlines,
	pdfstartview=FitH
}

\captionsetup{
    margin=20pt,
    font=small,
    labelfont=bf,
    labelsep=period
}

% Nuevos Comandos
\DeclarePairedDelimiter\abs{\lvert}{\rvert}
\renewcommand{\tablename}{Tabla}
\renewcommand{\figurename}{Figura}
\renewcommand{\contentsname}{Tabla de Contenido}
\renewcommand{\listtablename}{Lista de tablas}
\renewcommand{\listfigurename}{Lista de figuras}
\newcommand{\urlbib}[1]{{\footnotesize{\url{#1}}}} % para incluir urls en las referencias
\newcommand{\fullref}[1]{\ref{#1} de la página \pageref{#1}}


%% Ajustes
\setlength{\parskip}{0.5cm}
\decimalpoint