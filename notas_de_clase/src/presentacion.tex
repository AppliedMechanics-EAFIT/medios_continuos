%\chapter*{Presentación}
\section*{Presentación}

Gran parte de los problemas de Ingeniería se encuentran enmarcados en el correcto cálculo, determinación y análisis de la distribución de desplazamientos, tensiones y cambios de configuración para medios deformables sometidos a diferentes tipos de excitaciones.  La mecánica del medio continuo es el modelo matemático desarrollado a partir de principios físicos introducidos en la mecánica Newtoniana que permite resolver dicho problema.  En el mejor escenario posible y dependiendo del grado de complejidad de las geometrías de los medios a tratar, de las condiciones de frontera o del comportamiento mismo de los materiales constitutivos, se pueden lograr soluciones analíticas a dichos problemas.  Sin embargo, en las situaciones más complejas es necesario abordarlo mediante el uso de herramientas computacionales o experimentales.  En cualquier caso, el Ingeniero debe tener el suficiente conocimiento y dominio de las hipótesis básicas de los diferentes modelos matemáticos, así como de las ecuaciones que gobiernan el problema para poder así resolver, plantear, interpretar y analizar la amplia gama de posibles situaciones en el campo de la mecánica de medios continuos.

Estas notas de clase, desarrolladas por diversos docentes adscritos al Grupo de Investigación en Mecánica Aplicada, están orientadas a servir como herramienta de apoyo a los estudiantes del curso Mecánica de los Medios Continuos del pregrado en Ingeniería Civil de la Universidad EAFIT. Los contenidos de las mismas no son otra cosa que una recopilación más o menos concienzuda de material disponible en diferentes textos estructurado de manera que el material sea de fácil acceso a estudiantes con una formación física y matemática correspondiente al cuarto semestre de pregrado. Es importante aclarar que estas notas no tienen el rigor en cuanto a formato y redacción de un texto y que no buscan reemplazar la gran cantidad de excelentes contribuciones en Mecánica de los Medios Continuos disponibles en la literatura (ver sección de Bibliografía).

Las notas están organizadas de manera consistente con el curso cuya estructura y contenido se describe a continuación. 

\section*{Objetivos generales del curso}
Permitir al estudiante entender el modelo del medio continuo como una estrategia ingenieril de solución al problema central de la mecánica en un sistema de ``infinitas” partículas de manera que se genere un soporte físico-matemático lo suficientemente robusto para enfrentar problemas de la mecánica aplicada.


\section*{Objetivos específicos}

\begin{itemize}
%	\item Desarrollar el concepto matemático y físico de las n-ádicas, su generalización a tensores de orden n y la capacidad de hacer uso sistemático de la denominada notación indicial para plantear problemas de ingeniería y física.

	\item Desarrollar el concepto matemático y físico de un tensor de orden 2. 

	\item Identificar las hipótesis fundamentales y que de manera conjunta con el concepto del continuo matemático reducen los conceptos de fuerza y desplazamientos relativos entre partículas a los conceptos de tensión y deformación y al mismo tiempo re-visitar los conceptos de tensión y deformación y conocer algunas de las leyes constitutivas que gobiernan la relación entre estos.

	\item Reducir el problema central de la mecánica para un sistema de infinitas partículas, al de determinar los campos de desplazamientos, tensiones y deformaciones.
	
	\item Identificar matemáticamente el problema de la determinación de desplazamientos, tensiones y deformaciones en un medio continuo como un Problema de Valores en la Frontera (PVF).
\end{itemize}
%
\section*{Metodología}
%
Exposiciones magistrales, Elaboración de Ejemplos, Lecturas asignadas. El curso se divide en 2 partes.  En la parte I se revisan las hipótesis básicas que sustentan el modelo y se hace un breve repaso de álgebra vectorial. En la parte II se presentan los contenidos teóricos del modelo. En esta se parte de una revisión de los conceptos de tensión y deformación y obteniendo como resultado final las ecuaciones gobernantes o ecuaciones de campo.  Posteriormente se ligan las tensiones y deformaciones a través de un modelo constitutivo o ley de Hooke e introduce la hipótesis simplificadora de suponer el equilibrio en la configuración no-deformada del medio para llegar a la teoría linealizada de la elasticidad.  A la luz de esta teoría idealizada se estudian finalmente algunas soluciones analíticas o cerradas, discutiendo brevemente los métodos de solución y concentrándose más en el análisis y entendimiento de la solución misma.


\section*{Contenido de la materia}
\begin{itemize}
	\item[I.] Introducción
	\begin{itemize}
		\item[1.] Presentación del problema (1 semana).
		\begin{itemize}
			\item[1.1] Introducción y motivación del modelo del continuo. 
		\end{itemize}
	\end{itemize}

	\item[II.] FUNDAMENTOS TEÓRICOS
	\begin{itemize}
		\item[2.a] Análisis de tensiones. Parte 1 (4 semanas).
		\begin{itemize}
			\item[2.1] Concepto de tensión (Definición, Primer postulado de Cauchy, Definición de tensor).			
			\item[2.2] Ecuaciones de transformación y esfuerzos extremos en 2D y 3D. Círculo de Mohr
			\item[2.3] Solución de problemas
			\item Parcial 01.
		\end{itemize}			

		\item[2.b] Análisis de tensiones. Parte 2 (4 semanas).
		\begin{itemize}						
			\item[2.5] Ecuaciones de equilibrio en un medio continuo.
			\item[2.6] Estudio e interpretación de soluciones de tensiones.
			\item Parcial 02.
		\end{itemize}
		
		\item[3.] Análisis de Deformaciones (5 semanas).
		\begin{itemize}
			\item[3.1] Transformaciones lineales.
			\item[3.2] Definición de concepto de deformación.
			\item[3.3] Tensor de deformaciones.
			\item[3.4] Teoría de la Elasticidad. Ley de Hooke.
			\item[3.5] Problemas.
			\item Parcial 03.
		\end{itemize}
		
		\item[4.] Proyecto docente (2 semanas).
		\begin{itemize}
		\item[4.1] Estudio de solución numérica. 
		\item Examen Final.
		\end{itemize}
	\end{itemize}
\end{itemize}


