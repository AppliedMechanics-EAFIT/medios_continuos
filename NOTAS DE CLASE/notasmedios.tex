\documentclass[spanish,letterpaper,12pt,oneside,openany]{book}
\usepackage[spanish, activeacute]{babel}
\usepackage[utf8]{inputenc}
\decimalpoint
\usepackage{babelbib}
\usepackage{etoolbox}
\patchcmd\btxselectlanguage{\csname}{\csname TEMPPATCH}{}{} % Para hacer a babelbib funcionar
\usepackage{amstext, amssymb, amsthm, amsmath, amsbsy}
\usepackage{tabularx}
%\usepackage[ansinew]{inputenc}
\usepackage[margin=20pt,font=small,labelfont=bf,labelsep=period]{caption}%%Para modificar el formato del texto de los "caption"
\usepackage{graphicx}
\usepackage{cancel}
\usepackage{url}
\usepackage{listings}%%Para incluir codigo
\usepackage[final]{pdfpages}%%Para incluir archivos en pdf
\usepackage{geometry}
\usepackage{enumerate}
\usepackage{mathtools}									% Package to use abs command
\DeclarePairedDelimiter\abs{\lvert}{\rvert}			% abs command
\usepackage{subfig}%%Para incluir subgraficos
\usepackage{subfiles}														% include subfies	
\usepackage[pdftex, pdftitle={Notas de Clase: Mecánica de los Medios Continuos}, pdfauthor={Mecánica Aplicada}, pdfsubject={Class Notes}, pdfkeywords={Medios Continuos}, pdfpagemode=UseOutlines,bookmarks,bookmarksopen,pdfstartview=FitH,colorlinks,linkcolor=blue, urlcolor=black, citecolor=blue]{hyperref} %%Para incluir detalles cucas del pdf
\usepackage[ruled]{algorithm2e}%%Para incluir algoritmos
\usepackage{float}
\usepackage{multirow}
\usepackage{leftidx}
%\usepackage{algorithm2e}
\usepackage{algorithmic}
\usepackage{cite}  %% Para poner bonitas las citas
\usepackage{bookmark} %% Para poder organizar las etiquetas en pdf
\geometry{verbose,letterpaper,tmargin=3cm,bmargin=3cm,lmargin=2cm,rmargin=2cm}
%%%%%%%%%%%%%%%%%%%%%%%%%%%%%%%%%%%%%%%%%%%%%%%%%%%%%%%%%%%%%%%%%%%%%%%%%%%%
%Nuevos Comandos

\setlength{\parskip}{0.5cm}

% Comillas:   ``''

\newcommand{\urlbib}[1]{{\footnotesize{\url{#1}}}} % para incluir urls en las referencias
\newcommand{\fullref}[1]{\ref{#1} de la p\'agina \pageref{#1}}

\usepackage{cleveref}
\begin{document}
%%
\renewcommand{\tablename}{Tabla}
\renewcommand{\figurename}{Figura}
\renewcommand{\contentsname}{Tabla de Contenido}
\renewcommand{\listtablename}{Lista de tablas}
\renewcommand{\listfigurename}{Lista de figuras}

%%
%%%%%%%%%%%%%%%%%%%%%%%%%%%%%%%%%%%%%%%%%%%%%%%%%%%%%%%%%%%%%%%%%%%%%%%%%%%%
%%%%%%%%%% Portada %%%%%%%%%%
%Portada
%
\begin{center}
\vspace{50mm}
\Large{\textbf{Notas de Clase: Mec\'anica de los Medios Continuos}}
\large
\\[50mm]
Grupo de Investigación en Mecánica Aplicada
%\texttt{jgomezc1@eafit.edu.co}\\
%Cesar Sierra\\
%\texttt{casierra@gmail.com}
\\[70mm]
Grupo de Investigaci\'on en Mec\'anica Aplicada
\\
Departamento de Ingeniería Civil
\\
Escuela de Ingenierías
\\
Universidad EAFIT
\\
Medell\'in, Colombia
\\
\the\year
\end{center}
\thispagestyle{empty}

\frontmatter
%%%%%%%%%%%%%%%%%%%%%%%%%%%%%%%%%%%%%%%%%%%%%%%%%%%%%%%%%%%%%%%%%%%%%%%%%%%%
% Tabla de contenido
\pdfbookmark[0]{Tabla de Contenido}{}
\tableofcontents
\mainmatter
%%%%%%%%%%%%%%%%%%%%%%%%%%%%%%%%%%%%%%%%%%%%%%%%%%%%%%%%%%%%%%%%%%%%%%%%%%%%
\subfile{src/presentacion}
\subfile{src/algvect}
\subfile{src/Introduccion}
\subfile{src/tensiones}
\subfile{src/Equilibrio}
\subfile{src/deformaciones}
\subfile{src/elasticidad}
%%%%%%%%%%%%%%%%%%%%%%%%%%%%%%%%%%%%%%%%%%%%%%%%%%%%%%%%%%%%%%%%%%%%%%%%%%%%
%%%%%% Bibliografia %%%%%%%%%%
%%Para referenciar sin cita en el texto
%\nocite{}
\addcontentsline{toc}{chapter}{Referencias}
\bibliographystyle{babunsrt}
%\bibliography{informe}

%%%%%%%%%%%%%%%%%%%%%%%%%%%%%%%%%%%%%%%%%%%%%%%%%%%%%%%%%%%%%%%%%%%%%%%%%%%%

\end{document}

